% Resume for Jonathan Helmus
% Last updated July 6, 2015
%
\documentclass[margin,line]{res}


\oddsidemargin -.5in
\evensidemargin -.5in
\textwidth=6.0in
\itemsep=0in
\parsep=0in

\newenvironment{list1}{
  \begin{list}{\ding{113}}{%
      \setlength{\itemsep}{0in}
      \setlength{\parsep}{0in} \setlength{\parskip}{0in}
      \setlength{\topsep}{0in} \setlength{\partopsep}{0in} 
      \setlength{\leftmargin}{0.17in}}}{\end{list}}
\newenvironment{list2}{
  \begin{list}{$\bullet$}{%
      \setlength{\itemsep}{0in}
      \setlength{\parsep}{0in} \setlength{\parskip}{0in}
      \setlength{\topsep}{0in} \setlength{\partopsep}{0in} 
      \setlength{\leftmargin}{0.2in}}}{\end{list}}


\begin{document}

\name{Jonathan J. Helmus, Ph.D. \vspace*{.1in}}

\begin{resume}
\section{\sc Contact Information}


% Naperville Apartment Address
%\begin{tabular}{@{}p{2in}p{4in}}
%1928 Carlsbad Circle Apt 106 & {\it Phone:}  (331) 444-3896 \\            
%Naperville, IL 60563         & {\it E-mail:}  jjhelmus@gmail.com\\
%                             & {\it Code:} github.com/jjhelmus\\
%\end{tabular}

% Argonne Lab Address
\begin{tabular}{@{}p{2in}p{4in}}
Environmental Science Division  & {\it Phone (office):}  (630) 252-1488 \\            
Argonne National Laboratory     & {\it Phone (cell):}  (331) 444-3896\\
9700 South Cass Ave. Bldg 240   & {\it Email:} jjhelmus@gmail.com\\
Argonne, IL 60439               & {\it Code:} github.com/jjhelmus\\
\end{tabular}


\section{\sc Education}

{\bf University of Connecticut Health Center}, Farmington, Connecticut.
%{\em Department of Molecular, Microbial and Structural Biology} 
\begin{list1}
\item[] Postdoctoral Fellow, July, 2011-January 2013.
\item[] Advisor: Jeffrey Hoch.
%\item[] Research focus: Non-Fourier method for NMR data processing.
\end{list1}
\vspace{-0.1in}


{\bf The Ohio State University}, Columbus, Ohio.
%{\em Department of Chemical Physics} 
\begin{list1}
\item[] Ph.D., Chemical Physics, GPA 3.90, June 2005-July 2011.
\item[] Advisor:  Christopher Jaroniec
%\item[] Research focus: Solid State NMR on microcrystalline proteins, amyloid fibril and other biological solids.
\end{list1}
\vspace{-0.1in}

{\bf Coe College}, Cedar Rapids, Iowa.
\begin{list1}
\item[] NSF Research for Undergraduates (REU) student, Summer 2004.
%\item[] Research focus:  Synthesis and characterization of a varity of alkali vanadate glasses.
\end{list1}
\vspace{-0.1in}

{\bf Michigan Technological University}, Houghton, Michigan.
\begin{list1}
\item[] B.S. Chemistry (Chemical Physics), Minor: Mathematics, May 2005.
\end{list1}

\vspace{-0.1in}

\section{\sc Honors and Awards}
Presidential Fellowship from the Ohio State University, 2010-2011.\\
Phi Delta Gamma Graduate School Scholarship, 2010.\\
Isotech Experimental NMR Conference (ENC) Travel Scholarship, 2009, 2010.\\
ORAU sponsored attendee to the 59th Meeting of Nobel Laureates in Lindau, Germany, 2009.\\
ENC Student Travel Award, 2007, 2009, 2010.\\
Graduate School Fellowship, Ohio State University, 2005.\\
Michigan Tech. Physical Chemistry Student of the Year, 2003.\\
A.C.S. Upper Michigan Junior of the Year Majoring in Chemistry, 2003.\\
N.S. Mackie Scholarship, 2002.\\
%Deans List, 2001-2005.\\
Board of Control Scholarship, Michigan Tech., 2001-2005.\\
%Eagle Scout, BSA Troop 275, Kentwood, Michigan, 2001.\\
%Valedictorian of High School Class, 2001\\
%Finalist on National Chemistry Olympiad Test, 2000, 2001. \\

\vspace{-0.2in}
\section{\sc Research and Programming Experience}

{\bf Argonne National Laboratory}, Argonne, Illinois 
\hfill {\bf January 2013-present}\\
I am working in the environmental science division at 
Argonne National Laboratory as a scientist and software engineer 
for the Atmospheric Radiation Measurement (ARM) Climate Research Facility.
I am involved in a number of projects within ARM with other scientists and
developers to create and release value added products from data collected
by ARM’s scanning precipitation and cloud radars.  This work is highly 
collaborative with frequent input and interactions with scientists from 
universities and research centers across the globe.  In addition, 
I am the lead developer of the Python ARM Radar Toolkit (Py-ART),
an open source toolkit for the analysis and manipulation of weather radar data.

{\bf UConn Health Center}, Farmington, Connecticut
\hfill {\bf July 2011-January 2013}\\
I completed a postdoctoral position in the lab of Dr. Jeffrey C. Hoch 
where I worked on a number of projects focusing on using modern signal
processing methods to process NMR data. These projects include maintaining
and updating the Rowland NMR Toolkit, a software package for processing
NMR data written primarily in FORTRAN 77 and C.
In addition, I developed the software needed for my research into the 
effects of deconvolution kernels on NMR spectra processed using 
Maximum Entropy and $l_1$-norm (compressive sensing) reconstruction methods. 

%\vspace{-0.1in}

{\bf The Ohio State University}, Columbus, Ohio
\hfill {\bf June 2005-July 2011}\\
While performing research for my doctoral thesis, I worked on a
500 MHz Varian (Agilent) NMR spectrometer equipped with multiple probes
for magic angle spinning (MAS) of solid samples.  I shared responsibility for
the upkeep of the magnet with other graduate students including 
performing nitrogen and helium cryogen fills and troubleshooting hardware
and software problems. I wrote a number of pulse sequences for the 
spectrometer, as well as numerous scripts and software programs for
automated processing and analysis of solid state NMR data which are used 
by the entire Jaroniec research group.  In addition, I designed and wrote
the open source software package nmrglue (http://www.nmrglue.com) a 
Python module for working with NMR data.  
My major projects as a graduate student included structural and 
dynamic studies of the Y145Stop variant of the human prion protein and
development of methods for using covalently attached paramagnetic spin
labeled to probe distances in biological solids.

%\vspace{-0.1in}

{\bf Coe College}, Cedar Rapids, Iowa USA \hfill {\bf Summer 2004}\\
As a research experience for undergraduates students, 
I synthesised and characterized alkali vanadate glasses using plate and
roller quenching techniques. 

\section{\sc Software Engineering}
%{\bf NMR Software:} VnmrJ, NMRPipe, Sparky, SIMPSON, SPINEVOLUTION, nmrglue (author).\\
{\bf Programming Languages:} Python, FORTRAN (77 and modern versions), C, and C++.\\
{\bf Python modules:} NumPy, SciPy, matplotlib, IPython, Cython, and f2py\\
%{\bf Applications:} Linux text processing commands, GCC Compiler collection,
Microsoft Office suite, Adobe Illustrator, SigmaPlot.\\ 
{\bf Operating Systems:}  Linux, Windows, and OS X.\\ 

\section{\sc Publications}

16) M. Heistermann, S. Collis, M.J. Dixon, 
\underline{J.J. Helmus}, A. Henja, D.B. Michelson, Thomas Pfaff.
An Open Virtual Machine for Cross-Platform Weather Radar Science
{\em Bulletin of the American Meteorological Society} {\bf 2015}, 
early online release.

15) M. Heistermann, S. Collis, M.J. Dixon, S. Giangrande, 
\underline{J.J.  Helmus}, B. Kelley, J. Koistinen, D.B. Michelson, M. Peura, 
T. Pfaff, D.B. Wolff.
The Emergence of Open-Source Software for the Weather Radar Community,
{\em Bulletin of the American Meteorological Society} {\bf 2015}, 96, 117-128.

14) \underline{J.J. Helmus}, C.P. Jaroniec.  
Nmrglue: An open source Python package for the analysis of multidimensional NMR data,
{\em Journal of Biomolecular NMR} {\bf 2013}, 55, 355-367.

13) I. Sengupta, P.S. Nadaud, \underline{J.J. Helmus}, C.D. Schwieters, C.P. Jaroniec. 
Protein fold determined by paramagnetic magic-angle spinning solid-state NMR spectroscopy,
{\em Nature Chemistry} {\bf 2012}, 4, 410-417.

12) E.M. Jones, B. Wu, K. Surewicz, P.S. Nadaud, \underline{J.J. Helmus}, S. Chen, 
C.P. Jarnoniec, W.K. Surewicz. 
Structural polymorphism in amyloids: New insights from studies with Y145Stop prion protein fibrils.
{\em The Journal of Biological Chemistry} {\bf 2011}, 286, 42777-42784.

11) \underline{J.J. Helmus}, K. Surewicz, M.I. Apostol, W.K. Surewicz, C.P. Jaroniec. 
Intermolecular alignment in Y145Stop human prion protein amyloid fibrils probed by solid-state 
NMR spectroscopy, 
{\em Journal of the American Chemical Society} {\bf 2011}, 133, 13934-13937.

10) P.S. Nadaud, I. Sengupta, \underline{J.J. Helmus}, C.P. Jaroniec.
Evaluation of the influence of intermolecular electron-nucleus couplings and intrinsic metal 
binding sites on the measurement of $^{15}$N longitudinal paramagnetic relaxation enhancements 
in proteins by solid-state NMR. 
{\em Journal of Biomolecular NMR} {\bf 2011}, 51, 293-302.

9) P.S. Nadaud, \underline{J.J. Helmus}, I. Sengupta, C.P. Jaroniec. 
Rapid acquisition of multidimensional solid-state NMR spectra of proteins facilitated by 
covalently bound paramagnetic tags.
{\em Journal of the American Chemical Society} {\bf 2010}, 132, 9561-9563.

8) H. Shao, J. Seifert, N.C. Romano, M. Gao, \underline{J.J. Helmus}, C.P. Jaroniec, 
D.A. Modarelli, J.R. Parquette. 
Amphiphilic self-assembly of an n-type nanotube.
{\em Angewandte Chemie International Edition} {\bf 2010}, 49, 7688-7691. 

7) \underline{J.J. Helmus}, K. Surewicz, W.K. Surewicz, C.P. Jaroniec. 
Conformational flexibility of Y145Stop human prion protein amyloid fibrils probed by 
solid-state nuclear magnetic resonance spectroscopy. 
{\em Journal of the American Chemical Society} {\bf 2010} 132, 2393-2403.

6) P.S. Nadaud, \underline{J.J. Helmus}, S.L. Kall, C.P. Jaroniec. 
Paramagnetic ions enable tuning of nuclear relaxation rates and provide long-range structural 
restraints in solid-state NMR of proteins. 
{\em Journal of the American Chemical Society}. {\bf 2009}, 131, 8108-8120.

5) \underline{J.J. Helmus}, K. Surewicz, P.S. Nadaud, W.K. Surewicz, C.P. Jaroniec. 
Molecular conformation and dynamics of the Y145Stop variant of human prion protein in 
amyloid fibrils.
{\em Proceedings of the National Academy of Sciences USA}, {\bf 2008}, 105, 6284-6289.

4) \underline{J.J. Helmus}, P.S. Nadaud, N. H\"{o}fer, C.P. Jaroniec. 
Determination of methyl $^{13}$C-$^{15}$N dipolar couplings in peptides and proteins by 
three-dimensional and four-dimensional magic-angle spinning solid-state NMR spectroscopy.
{\em Journal of  Chemical Physics}, {\bf 2008}, 128, 052314.

3) P.S. Nadaud, \underline{J.J. Helmus}, C.P. Jaroniec. 
$^{13}$C and $^{15}$N chemical shift assignments and secondary structure of the 
B3 immunoglobulin-binding domain of streptococcal protein G by magic-angle spinning 
solid-state NMR spectroscopy.
{\em Biomolecular NMR Assignments}, {\bf 2007}, 1, 117-120. 

2) P.S. Nadaud, \underline{J.J. Helmus}, N. H\"{o}fer, C.P. Jaroniec. 
Long-range structural restraints in spin-labeled proteins probed by solid-state nuclear magnetic 
resonance spectroscopy.
{\em Journal American Chemical Soceity}, {\bf 2007}, 129, 7502-7503.

1) S. Giri, C. Gaebler, \underline{J. Helmus}, M. Affatigato; S. Feller, M. A. Kodama.
General study of packing in oxide glass systems containing alkali.
{\em Journal of Non-Crystalline Solids}, {\bf 2004}, 347, 87-92.

% lite version
\section{\sc Professional Meeting - Poster Presentations}
2015 ARM/ASR Joint User Facility PI Meeting, Vienna, VA (March 16-12)\\
SciPy 2014 Conference, Austin, TX (July 6-12, 2014)\\
36th Conference on Radar Meteorology, Breckenridge, CO, (September 16-20, 2013)\\
SciPy 2013 Conference, Austin, TX (June 24-29, 2013)\\
2013 ASR Science Team Meeting, Potomac, MD (March 18-21, 2013)\\
52nd Experimental NMR Conference, Pacific Grove, CA (April 10-15, 2011) \\
51st Experimental NMR Conference, Daytona Beach, FL (April 18-23, 2010)\\ 
50th Experimental NMR Conference, Pacific Grove, CA (March 29-April 3, 2009)\\
50th Rocky Mountain Conference for Analytical Chemistry, Breckenridge, CO (July 27-31, 2008)\\ 
ACS 39th Central Regional Meeting, Columbus, OH (June 10-14, 2008)\\
Pittsburgh NMR Symposium, Pittsburgh, PA (April 29-30, 2008)\\
49th Experimental NMR Conference, Pacific Grove, CA (March 9-14, 2008)\\
48th Experimental NMR Conference, Daytona Beach, FL (April 22-27, 2007)\\
Research Coordination Network Workshop, Bethesda, MD (January 12-13, 2007)\\

\section{\sc Professional Meeting - Attended}
PyCon 2014, Montreal, Canada (April 9-17)\\
SEA Software Engineering Conference 2013, Boulder, CO (April 1-5, 2013)\\
59th Meeting of Nobel Laureates in Lindau, Lindau, Germany (June 28-July 3, 2009)\\
U.S.-Canada Winter School on Biomolecular Solid State NMR, Stowe, VT (January 20-25, 2008)\\

\section{\sc Invited Talks}

{\bf Exploring Open Access Weather Radar with the Python ARM Toolkit},
Scipy 2015 Conference, Austin, TX, July, 10, 2015.

{\bf Speeding Up Python Data Analysis Using Cython},
DePy 2015, Chicago, IL, May 29, 2015.

{\bf Profiling Python code to improve memory usage and execution time},
2015 SEA Software Enginnering Conference. Boulder, CO, April 14, 2015.

{\bf Designing and implementing radar algorithms in Python},
95th AMS Annual Meeting, Phoenix, AZ, January 5, 2014.

{\bf New Doppler Spectral Processing Technique for Identifying 
Atmospheric Signals from Radar Wind Profilers},
8th European Conference on Radar in Meteorology and Hydrology, 
Garmisch-Partenkirchen, Germany, September 4, 2014.

{\bf Tools and Techniques for Developing Atmospheric Python Software: 
Insight from the Python ARM Radar Toolkit}, 
94th AMS Annual Meeting, Atlanta, GA, February 3, 2014.

{\bf The Development and Use of a Python Data Analysis Package: Nmrglue},
Argonne National Laboratory, Argonne, IL, October 15, 2012.

{\bf nmrglue: a Python Module for Working with NMR Data}, 
Scipy 2012 Conference, Austin, TX, July, 19, 2012.

{\bf Rapid Acquisition of Multidimensional Solid State NMR Spectra for the Study of 
Proteins}, 
Physical Chemistry Student Lecture Series, The Ohio State University, Columbus, OH, Jan. 12, 2010.

{\bf Structure and Dynamics of Proteins using Solid State Nuclear Magnetic Resonance 
Spectroscopy},
Physical Chemistry Student Lecture Series, The Ohio State University, Columbus, OH, Nov. 6, 2007.

\end{resume}
\end{document}
